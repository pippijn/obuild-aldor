\clearpage
\section{Connecting to the characteristic set library from within Mathematica}
\defsec{casmathematica}

This section illustrates how the exported code of \refsec{preparationsinaldor} can be connected to \Mathematica.

The first part of this section analyzes \Mathematica and \MathLink, the framework to connect code to \Mathematica. The second part discusses how \MathLink enabled executables are generated. This second part heavily relies on template files which are described in depth in the last part of this section.

\subsection{A closer look at Mathematica}

Before connecting \exportedsymbol to \Mathematica, it is important to investigate \Mathematica closer and identify its sub-parts. Afterwards a sub-part to connect \exportedsymbol to is identified and the used technology is presented.

The computer algebra system \Mathematica\cite{Mathematica} provides two main components, which are 
\begin{itemize}
\item the notebook interface and 
\item the kernel.
\end{itemize}
While the notebook interface is typically used to input expressions and receive output, all computations are performed by the kernel. This splitting up is not artificial, but can be seen already in the file system. The executable \file{math} represents \Mathematica's kernel, \file{mathematica} represents \Mathematica's notebook interface. These two programs are independent from each other. The notebook interface can be used without the kernel and vice versa. However, the notebook interface has to be connected to a kernel for performing computations\footnote{On evaluating the first cell within a notebook, the notebook interface typically starts a kernel locally. The user is given no feedback about this, except that evaluating the first cell takes considerably longer than subsequent evaluation of cells. This delay is due to starting a local kernel. It is not necessary to start a local kernel. The notebook interface also allows to use kernels running on other computers.}. As the computational engine of \Mathematica is its kernel, \exportedsymbol has to be connected to the kernel. \Mathematica provides an interface to its kernel via \MathLink. \MathLink\cite{MathLink} is \Mathematica's standard to pass data between applications. For example, the connection between the notebook interface and a kernel is made through \MathLink.

The rest of \refsec{casmathematica} discusses how \exportedsymbol is connected to \Mathematica's kernel via \MathLink.

It is important to see that \MathLink is not a further program, but simply a name for the specification of how to represent and transmit data. \Mathematica's kernel does not allow to import libraries. It can only import code via \MathLink.

\MathLink is \Mathematica's proposed way to connect foreign code to the \Mathematica kernel. Foreign code has to implement the \MathLink functionality to be able to connect to the kernel via \MathLink. For common languages like \C, or \Java software development kits are provided. These kits can be used to built \MathLink enabled programs.

As \exportedsymbol follows \C calling conventions, the \C software development kit is used for this discussion. Although this kit is shipped with \Mathematica, it can be obtained independently from \Mathematica at \cite{MathLinkSDK}.

For this treatment, \MathLink's \C software development kit that is shipped with \Mathematica $5.1$ is used. This version's documentation of the contained \Mcc script does not match the \Mcc script itself. This fact also holds for the other versions of the software development kit that are available to the author. These are those of \Mathematica $3.0.2$, \Mathematica $4.0$, \Mathematica $4.1$, \Mathematica $4.2$, \Mathematica $5.0$, and the kit from \cite{MathLinkSDK} on $15^{th}$ October $2005$. However, the version from \Mathematica $5.1$ is chosen.

The main parts of the \C software development kit for \Mathematica are
\begin{itemize}
\item \Mcc,
\item \Mprep,
\item \file{mathlink.h}, and
\item \file{libML.a}.
\end{itemize}

\Mcc is a script automating compilation of \MathLink template files to \MathLink enabled executables. \MathLink template files are treated separately in \refsec{mathlink}.

\Mprep is a preprocessor converting \MathLink templates to \C files.

\file{mathlink.h} is a \C header source file. This file is needed to compile the \C files generated by \Mprep.

\file{libML.a} is a library implementing the definitions of \file{mathlink.h}. This file is needed to link object files to executables.

We investigate \Mcc further, as it hides the use of the other three parts from the user. With the help of \Mcc, a \MathLink enabled executable is built that can easily be connected to the \Mathematica kernel.

\subsection{MathLink's Mcc compiler script and MathLink enabled executables}
\defsec{mcc}

This section focuses on \Mcc and its use to compile \Aldor's exported \C code to \MathLink enabled executables. Additionally, at the end of this section it is shown how to connect \MathLink enabled executables to \Mathematica.

\Mcc is a wrapper for the \C compiler and converts \MathLink template files to \MathLink enabled executable files. \MathLink template files specify how and which \C functions are to be made accessible through \MathLink. These template files are discussed in detail in \refsec{mathlink}.

Internally, the \Mcc script operates in two stages. The first stage passes the \MathLink template files to \Mprep. \Mprep is a preprocessor that converts \MathLink templates to a single \C file. This \C file depends on \file{mathlink.h}. In the second stage of \Mcc, the generated \C file from \Mprep is compiled and linked against the library \file{libML.a}. The work-flow for \Mcc is depicted in \reffig{mccmprep}.

\diagrameps[7cm]{mccmprep}{The internals of \Mcc}

In the further discussion, command line options to the \Mcc script are presented. These command line options refer to the script itself. These options' implementation sometimes does not match their documentation. Other options are completely undocumented. However, as mentioned above, the documentation of \Mcc that is shipped with the used software development kit is not accurate. Therefore, the documentation is ignored and the script itself is analyzed. Furthermore, \Mcc relies on a \C compiler. As \Mcc passes several options directly to the \C compiler, it is important to use a compatible \C compiler. It is assumed, that \Gcc is used as \C compiler for \Mcc{}\footnote{\Mcc tries to use the \C compiler specified by the environment variable \environmentvariable{CC}. If \environmentvariable{CC} is empty, \Mcc falls back to using \file{cc}. On most \GNULinux systems, \environmentvariable{CC} is empty and \file{cc} refers to \Gcc. Therefore, on most \GNULinux systems, \Mcc defaults to using \Gcc.}. Additionally, let \commandline{MLINCDIR} denote the directory of \file{mathlink.h} and \commandline{MLLIBDIR} the directory of \file{libML.a}. Typically both directories refer to the \file{AddOns/MathLink/DeveloperKit/Linux/CompilerAdditions} sub-directory of \Mathematica. 

When adapting \C code for the use with \MathLink, it usually suffices to incorporate the implemented \C code into a template file. Assuming this template file is called \file{sometemplate.tm}, a call to

\commandline{mcc sometemplate.tm}

compiles it to a \MathLink executable. This executable is called \file{a.out}. \Mcc's command line option \commandlineparameter{-o} can be used to override the output file.

\commandline{mcc -o someexecutable sometemplate.tm}

compiles \file{sometemplate.tm} to the executable \file{someexecutable}. Such a call however does not work for the \file{charset.tm} \MathLink template file of \refsec{mathlink}. That \MathLink template has to be linked with the implementation of \exportedsymbol. This implementation can be found in the \LibCharSet library. Therefore, \Mcc has to instruct the underlying compiler to link against the \LibCharSet library\footnote{When compiling to an executable, it does not suffice, to link against \LibCharSet, as \LibCharSet again has unresolved symbols. These symbols are resolved in \LibExtIO, \LibAlgebra, \LibAldor. These libraries again have unresolved symbols. In the end, the \MathLink file has to be linked against the libraries \LibCharSet, \LibExtIO, \LibAlgebra, \LibAldor, \LibFoam, and \LibM. Specifying all these libraries for every upcoming command line renders the given commands unreadable. Therefore, only the first of these libraries is given, as the \LibCharSet library suffices to illustrate the problems.}. The necessary option for this linkage is \commandlineparameter{-lcharset}. However, when calling \Mcc by

\commandline{mcc -o charset charset.tm -lcharset}

\Mcc invokes the \C compiler with

\begin{verbatim}
gcc -o charset -I MLINCDIR -lcharset charset.tm.c \
  -L MLLIBDIR -lML -lm
\end{verbatim}

while the correct call would be 

\begin{verbatim}
gcc -o charset -I MLINCDIR charset.tm.c -lcharset \
  -L MLLIBDIR -lML -lm
\end{verbatim}

It is however crucial that \commandlineparameter{charset.tm.c} comes \emph{before} \commandlineparameter{-lcharset}. Otherwise, the compiled \file{charset.tm.c} is not linked against the \LibCharSet library\footnote{This behaviour of \Gcc is not a bug. \Gcc resolves the symbols of object files from left to right. Therefore, only the object files of \file{libML.a} and \file{libm.a} are used to resolve the symbols of \file{charset.tm.c}'s object file when using \Mcc. \Ld, which is used by \Gcc for linking, can be set up to use groups of libraries. This grouping of libraries allows to pass \file{charset.tm.c} after \file{-lcharset} and still have all symbols resolved. For example\\
\commandline{gcc -o charset -I MLINCDIR -Xlinker --start-group charset.tm.c -lcharset $\backslash$}\\
\hspace{2cm}\commandline{-Xlinker --end-group -L MLLIBDIR -lML -lm}\\
groups \LibCharSet and the object file of \file{charset.tm.c}. Although this call to \Gcc works and the call states \commandline{-lcharset} after \file{charset.tm.c}, it cannot be used to solve the \Mcc issues. With this solution \commandline{-lcharset} can be specified after \file{charset.tm.c}. However, \commandline{-Xlinker --start-group} has to occur \emph{before} \file{charset.tm.c}. Therefore, an equivalent problems arise, as\\
\commandline{mcc -o charset -Xlinker --start-group charset.tm -lcharset -Xlinker  $\backslash$}\\
$\quad$\hspace{2cm}\commandline{ --end-group}\\
resolves to\\
\commandline{gcc -o charset -I MLINCDIR -Xlinker --start-group -lcharset $\backslash$}\\
\hspace{2cm}\commandline{-Xlinker --end-group charset.tm.c -L MLLIBDIR -lML -lm}}.

The possible workarounds are
\begin{itemize}
\item to call \Mprep and \Gcc manually,
\item to use \Mcc only to compile the file and link the file manually later on, or
\item to extract and modify the call to the \C compiler.
\end{itemize}

Calling \Mprep and \Gcc manually overcomes the need for \Mcc. The commands
\begin{verbatim}
mprep -o charset.tm.c charset.tm
gcc -o charset -I MLINCDIR charset.tm.c -lcharset \
  -L MLLIBDIR -lML -lm
\end{verbatim}
compile the \file{charset.tm} template to the \C file \file{charset.tm.c}, which is in turn compiled to the executable \file{charset}. This solution is the most flexible of the three workarounds. Every command can be called with exactly those options that are required. The major disadvantage of this workaround is that it does not use \Mcc at all. However, \Mcc is the \MathLink's proposed way to compile \MathLink enabled executables. Therefore, future versions of \MathLink will focus on keeping the use of \Mcc consistent. The use of \Mprep is subject to changes. As a result, it can be expected that the presented use of \Mprep is obsolete in further versions of the \MathLink \C software development kit. Furthermore, the corresponding \commandline{MLINCDIR} and \commandline{MLLIBDIR} have to be determined manually. When using \Mcc to build the executable, \Mcc locates the proper directories on its own.

The second workaround is to use \Mcc only to compile the file and link manually. The first line of the command sequence
\begin{verbatim}
mcc -c charset.tm
gcc -o charset charset.tm.o -lcharset -L MLLIBDIR -lML -lm
\end{verbatim}
uses \Mcc to compile the template \file{charset.tm} to the object file \file{charset.tm.o}. The command line option \commandlineparameter{-c} tells \Mcc to use \Mprep to generate the \C file \file{charset.tm.c} and finally use the \C compiler to compile the \C file to the object file \file{charset.tm.o}. In the second line, this object file is linked to an executable by \Gcc. It is also possible to use \Ld, the GNU Linker, for linking the object file. However, using \Ld requires to pass further options and link against further libraries. When using \Gcc, these issues are hidden from the user. 

The presented workaround does no longer invoke \Mprep explicitly and does not require to pass the directory \file{MLINCDIR} to any command. However, \file{MLLIBDIR} is still required. Furthermore, future versions of \MathLink may come with additional or other libraries and may require to link against some of these libraries. As a result, the call to gcc has to be adjusted. Therefore, the presented workaround is again likely to fail for future versions of \MathLink's \C software development kit.

The last workaround is to extract and modify the call to the \C compiler. The commands
\begin{verbatim}
export OLDCC=$CC 
if [ -z "$OLDCC" ] ; then export OLDCC=cc ; fi
export CC=echo
$OLDCC `mcc -g -o charset charset.tm` -lcharset
export CC=$OLDCC
\end{verbatim}
can be used in any \Bash-like shell. The first three lines are used to store the system's \C compiler in the environment variable \environmentvariable{OLDCC} and set the environment variable \environmentvariable{CC} to the \commandline{echo} command. As \Mcc uses \environmentvariable{CC} as \C compiler, \Mcc's calls to the \C compiler (see line $4$) are echoed. The command \commandline{`mcc -g -o charset charset.tm`} evaluates to 
\begin{verbatim}
-o charset.mathlink -I MLINCDIR -g charset.tm.c -L MLLIBDIR -lML -lm
\end{verbatim}
which are the arguments to the \C compiler to compile to an executable. The forth line of this workaround prepends the \C compilers executable and appends the \commandlineparameter{-lcharset} to link against the \LibCharSet library. Thereby, the \C compiler is invoked with the proper options and the appended \commandlineparameter{-lcharset}. The last line of the workaround resets the environment variable \environmentvariable{CC}.

This workaround passes the command line parameter \commandlineparameter{-g} to \Mcc. 
After compilation, \Mcc typically removes the generated \C and object files. For this workaround, it is essential that \Mcc does \emph{not} remove the generated \C file, as this file is needed for the \C compiler. Therefore, \Mcc is called with the parameter \commandlineparameter{-g}. This parameter tells \Mcc to not remove the generated \C file. Additionally, \Mcc passes the \commandlineparameter{-g} to the \C compiler. \C compilers generate debugging information when passed the \commandline{-g} option.

This workaround is the most complex of the three presented alternatives. However, it does not require to specify \commandlineparameter{MLLIBDIR} or \commandlineparameter{MLINCDIR}. Therefore, it is the only workaround that works without further knowledge of the used system. Additionally, it is the only workaround that can be expected to work in future versions of \MathLink, as it does not call \Mprep directly and makes no assumption on which libraries to use. Therefore, the third workaround is proposed among the three presented ones.

%With each of the three presented workarounds a \MathLink enabled executable \file{charset} can be generated from the \MathLink template of \refsec{extendedmathlink}. To use this executable from within \Mathematica it suffices to call
Each of the three presented workarounds can be used to generate a \MathLink enabled executable \file{charset} the \MathLink template of \refsec{extendedmathlink}. To use this executable from within \Mathematica it suffices to call

\begin{mathematicaprogram}
Install["/path/to/charset"]
\end{mathematicaprogram}

in either the notebook interface or the kernel itself, where \file{/path/to} denotes the directory of \file{charset}. \mathematicacode{Install} starts the \file{charset} executable and connects the kernel to it. Afterwards the function \mathematicacode{CoherentAutoreducedSet} of \file{charset} can be used. There is no need to explicitly bring the package \mathematicacode{CharacteristicSet} into scope.

%\MathLink also allows the generated executable to run on a different computer as the \Mathematica kernel. This does not require any modifications of \file{charset}, \MathLink also allows 

%The \mathematicacode{Install} command starts the program on the computer on which the kernel is running. However, it is possible to execute the kernel and the \file{charset} \MathLink executable on different computers. In such a setting, it is necessary to start the \MathLink executable manually on a computer with the \commandlineparameter{-linkcreate}. If this is successful, the executableThere the executable gives prompts for a link. 



%\begin{mathematicaprogram}
%Mathematica 5.1 for Linux
%Copyright 1988-2004 Wolfram Research, Inc.
% -- Motif graphics initialized -- 
%
%In[1]:= Install["charset.mathlink"]
%
%Out[1]= LinkObject[./charset.mathlink, 1, 1]
%
%In[2]:= depvars := { y1, y2, y3 }
%
%In[3]:= indepvars := { s, t }
%
%In[4]:= poly1 := D[y2[s,t],t]*y3[s,t]
%
%In[5]:= poly2 := D[y1[s,t],s]
%
%In[6]:= poly3 := y2[s,t]
%
%In[7]:= polys := { poly1, poly2, poly3 }
%
%In[8]:= domOptions := ""
%
%In[9]:= CoherentAutoreducedSet[ polys, depvars, indepvars, domOptions ]
%
%                     (1,0)
%Out[9]= {y2[s, t], y1     [s, t]}
%\end{mathematicaprogram}

\subsection{MathLink template files}
\defsec{mathlink}

\refsec[Section]{mcc} used \MathLink template files without specifying what \MathLink template files are. The current section gives a detailed discussion of these template files.

The first part of this section gives the syntax for a \MathLink template file and provides a basic example for \exportedsymbol. In the second part of this section, this basic example is refined and better integrated into \Mathematica.

\subsubsection{A basic MathLink template file}
\defsec{basicmathlink}

\MathLink template files are used to describe \C functions to make available through \MathLink. These template files may contain \C source code and \MathLink commands. Lines not beginning with a colon are treated as \C code. Lines beginning with a colon are treated as \MathLink commands. Such lines are of the form colon, command name, colon optionally followed by arguments to the command. \MathLink provides eight different commands. 
\begin{itemize}
\item \mathematicacode{Begin} is used to start the definition of a function that is to be made accessible through \MathLink. This command takes no options. 
\item \mathematicacode{Function} takes one parameter, which is the name of the \C function to make accessible. 
\item \mathematicacode{Pattern} is used to define the function's \MathLink name. The \MathLink name and the \C name of a function need not be identical. \mathematicacode{Pattern} takes one argument, which is the \Mathematica calling pattern for the function. 
\item \mathematicacode{Arguments} takes one argument which is the list of arguments to pass to the \C function. This list may contain arbitrary \Mathematica expressions, for example \mathematicacode{If} constructs. Additionally, these expressions may contain the parameters of the \mathematicacode{Pattern} command. The expressions of \mathematicacode{Arguments} are evaluated for every function call of the \MathLink function.
\item \mathematicacode{ArgumentTypes} denotes the corresponding types for the arguments specified by \mathematicacode{Arguments}.
\item \mathematicacode{ReturnType}'s argument specifies the return type of the \C function. 
\item \mathematicacode{End} is used to end the definition of a function that is to be made accessible through \MathLink. 
\item \mathematicacode{Evaluate} allows to execute arbitrary \Mathematica commands. However, the command \mathematicacode{Evaluate} is discussed in more detail in \refsec{extendedmathlink}, as it is not necessary for making \C functions accessible via \MathLink.
\end{itemize}

With the first seven commands the mapping between the \C function and the \MathLink function can be described completely.

\begin{itemize}
\item \mathematicacode{Begin} and \mathematicacode{End} are used for scoping.
\item \mathematicacode{Function}, \mathematicacode{ArgumentTypes}, and \mathematicacode{ReturnType} describe the \C function.
\item \mathematicacode{Pattern} and \mathematicacode{Arguments} describe the \MathLink function.
\end{itemize}

An example template specification for \exportedsymbol is given by the following piece of code. The argument of \mathematicacode{Pattern} is too long to fit on the page. Therefore, it is broken up into two lines. Breaking lines with \MathLink commands is not allowed in \MathLink template files and renders the given code an invalid \MathLink template file. Nevertheless, the line is kept broken into two lines and has to be interpreted as if these two line were actually one line. For \refsec{basicmathlink} and \refsec{extendedmathlink}, \mathematicacode{--} at the end of a line marks a line break that must not occur in the \MathLink template file, but is necessary for formatting the code in this documentation.

\defpage{tmexample}
\begin{mathematicaprogram}
:Begin:
:Function:      _charset_coherentAutoreducedSet
:Pattern:       CoherentAutoreducedSetAldor[ langStr_String, polyStr_String,      --
  varStr_String, depStr_String, optStr_String ]
:Arguments:     { langStr, polyStr, varStr, depStr, optStr }
:ArgumentTypes: { String, String, String, String, String }
:ReturnType:    String
:End:
\end{mathematicaprogram}

In addition to the \MathLink definition, a \MathLink template has to contain an entrance point for \MathLink and an implementation of the \C function that is referenced by \MathLink's \mathematicacode{Function} command. The entry point for \Mathematica is set up by calling the \MathLink's \C function \ccode{MLMain}. This call can be set up with the following piece of code.

\begin{mathematicaprogram}
int main(int argc, char* argv[])
{
  return MLMain(argc, argv);
}
\end{mathematicaprogram}

As mentioned before, the \MathLink template also has to provide the \C function which is made accessible via \MathLink. For this discussion, the implementation of \exportedsymbol is hidden within the \LibCharSet library and cannot be given in the template file. However, \C allows to defer the implementation to a library by the keyword \ccode{extern}. The following \C code denotes that the implementation of the function \exportedsymbol is not contained in this file, but can expected to be found in a library later on.

\begin{mathematicaprogram}
extern char * _charset_coherentAutoreducedSet( kcharp_ct langStr, kcharp_ct polyStr, 
  kcharp_ct varStr, kcharp_ct depStr, kcharp_ct optStr);
\end{mathematicaprogram}

In \Aldor (see \refsec{exportingaldorsfunctions}), \exportedsymbol has been exported as function taking five pointers and returning a pointer. Although each of the pointers is expected to point to a string, this cannot be seen from \exportedsymbol's definition in \Aldor. This knowledge is implicit. However, the given piece of code denotes that \exportedsymbol takes five arguments of \ccode{kcharp_ct} and returns a \ccode{char *}. Both, \ccode{char *} and \ccode{kcharp_ct} denote pointers to strings in \C and are discussed later. Although \C allows to model arbitrary pointers\footnote{In \C, pointers to arbitrary entities are modelled by \ccode{void *}.} which correspond to \Aldor's \adtype{Pointer}, the given piece of code uses pointers to strings. These types are consequences of the \MathLink template on \refpage{tmexample}. There, \mathematicacode{String}{}s are used as types for \mathematicacode{ArgumentTypes} and \mathematicacode{ReturnType}. In this template, the implicit knowledge that the pointers actually point to strings is made explicit. However, it is not possible to have arbitrary pointers in \mathematicacode{ArgumentTypes} or \mathematicacode{ReturnType}, as \Mathematica does not provide the concept of arbitrary pointers. Therefore, it is crucial for \MathLink to know what kind of pointers a function is dealing with.

The \MathLink template file uses the types \ccode{char *} and \ccode{kcharp_ct} for pointers to strings. In \C, \ccode{char} is used as type for a character and \ccode{*} denotes a pointer to a type. Therefore, \ccode{char *} is actually a pointer to a character. However, in \C strings are typically represented by a pointer to the first character of a string. All subsequent characters and a trailing zero byte are stored in the consecutive bytes. Therefore, there is no syntactical difference between a pointer to a character and a pointer to a string in \C. \ccode{kcharp_ct} is an alias of \MathLink to \C's \ccode{const char *}\footnote{This statement holds for operating systems like \GNULinux on \xeightysix. However, the exact expansion of \ccode{kcharp_ct} can be found in \file{mathlink.h} and is \ccode{MLCONST char FAR *}. The definition of \ccode{MLCONST} is far too complicated to be reproduced in this documentation and is again found in \ccode{mathlink.h}. However, in typical applications on up-to-date operating systems and \C compilers, it is save to assume \ccode{MLCONST} to evaluate to \ccode{const}. \ccode{FAR} evaluates to \ccode{far} on operating systems that use segmented pointers to memory, like \MSDOS. There, \ccode{far} has to be used for pointers to memory in other segments than the current one. On other operating systems like \GNULinux, \ccode{FAR} is blank.}. \ccode{const char *} is a pointer to a string, whose value cannot be changed\footnote{\ccode{const char *} strings cannot be modified directly. However it is possible to discard the \ccode{const} qualifier (for example by casting to \ccode{char *}) and modify the string afterwards. Although possible, it is considered bad style to discard qualifiers. \C compilers typically issue warnings, when qualifiers are removed from a type.}.

The complete \MathLink template file is given by the following piece of code.

\begin{mathematicaprogram}
:Begin:
:Function:      _charset_coherentAutoreducedSet
:Pattern:       CoherentAutoreducedSetAldor[ langStr_String, polyStr_String,      --
  varStr_String, depStr_String, optStr_String ]
:Arguments:     { langStr, polyStr, varStr, depStr, optStr }
:ArgumentTypes: { String, String, String, String, String }
:ReturnType:    String
:End:

extern char * _charset_coherentAutoreducedSet( kcharp_ct langStr, kcharp_ct polyStr, 
  kcharp_ct varStr, kcharp_ct depStr, kcharp_ct optStr);

int main(int argc, char* argv[])
{
  return MLMain(argc, argv);
}
\end{mathematicaprogram}

\subsubsection{Advanced topics of MathLink templates}
\defsec{extendedmathlink}

The template file of \refsec{basicmathlink} allows \exportedsymbol to be called directly from \Mathematica. When calling \exportedsymbol directly, the language parameter has to be specified for every call. Furthermore, it is necessary to convert the parameters to strings and the result to a \Mathematica expression for every call. Although \Mathematica allows to attach short help texts to functions, the \MathLink template of \refsec{basicmathlink} does not use this feature. 

This section overcomes these issues by \MathLink's \mathematicacode{Evaluate} command. After discussing each of these issues separately, a complete \MathLink file is given that incorporates the presented changes.

\mathematicacode{Evaluate} allows to execute arbitrary \Mathematica commands upon connection to the \Mathematica kernel. The following piece of code defines a wrapper for the function \mathematicacode{CoherentAutoreducedSetAldor}. This wrapper is called \mathematicacode{CoherentAutoreducedSet} and takes care of the conversion between strings and \Mathematica expressions. Additionally, the used language is automatically set to \mathematicacode{mathematicafullform}.

\begin{mathematicaprogram}
:Evaluate:      CoherentAutoreducedSet[ poly_List, var_List, dep_List,            --
    opts_String:"" ] := ToExpression[ CoherentAutoreducedSetAldor[                --
    "mathematicafullform", ToString[ poly//FullForm ], ToString[ var//FullForm ], --
    ToString[ dep//FullForm ], ToString[ opts//FullForm ] ] ]
\end{mathematicaprogram}

The function \mathematicacode{ToExpression} converts a string to a \Mathematica expression. \mathematicacode{FullForm} formats an expression in full form syntax. However, the result of \mathematicacode{FullForm} is not a string but again an expression. Therefore, \Mathematica's \mathematicacode{ToString} is used to convert the full form expressions to strings. %Nevertheless, \mathematicacode{FullForm} is important to convert the \Mathematica expressions to \Mathematica expresions in full form syntax.

A help text is attached to a function by setting its \mathematicacode{usage} message. The following code gives an example for \mathematicacode{CoherentAutoreducedSet}.


\begin{mathematicaprogram}
:Evaluate:      CoherentAutoreducedSet::usage = "CoherentAutoreducedSet[          --
    { a, b, ...}, {y1,y2,...}, { s1, s2, ... }, Options ] derives a coherent      --
    autoreduced set of the polynomials a, b, ... in the dependent variables y1,   --
    y2,  ... and independent variables s1, s2, ... . Options is an optional       --
    string denoting options as described in the reference manual for the          --
    characteristic set library of Aldor."
\end{mathematicaprogram}

Finally, the code is put into a package by the the following pattern. 

\begin{mathematicaprogram}
:Evaluate:      BeginPackage["PackageName`"]

    public definitions

:Evaluate:      Begin["`Private`"]

    private definitions

:Evaluate:      End[ ]
:Evaluate:      EndPackage[ ]
\end{mathematicaprogram}

In this pattern, \mathematicacode{public definitions} is to be replaced with the definitions that should be publicly available and accessible after connection to a kernel.
\mathematicacode{private definitions} is to be replaced by definitions that are not intended for public use. Although this pattern is proposed in \MathLink's documentation, \Mathematica does allow to hide a function completely. The definitions in the \mathematicacode{Private} part can be accessed, when completely qualifying their name. Although, the function \mathematicacode{CoherentAutoreducedSetAldor} of the \MathLink template file on \refpage{completemltemplate} is defined within a \mathematicacode{Private} part, 

\begin{mathematicaprogram}
`CharacteristicSet`Private`CoherentAutoreducedSet
\end{mathematicaprogram}

can be used to access the function.

The \MathLink template on \refpage{completemltemplate} adds the improvements of this section to the basic template of \refsec{basicmathlink}.

In \refsec{casmathematica} only the \MathLink template commands of the \MathLink's \C software development kit have been used. These commands are sufficient for connecting \exportedsymbol code to \Mathematica. However, the \MathLink \C software development kit also provides several \C functions to be used in \C code. There are, for example, functions to access \Mathematica's internal data structures for expressions. These functions are documented in the manual of the \C software development kit and allow to convert custom data types to \Mathematica expressions and vice versa. Since \exportedsymbol takes string parameters, these functions need not be used and conversion between \Aldor and \Mathematica expressions is left to \Mathematica's \mathematicacode{ToString}, \mathematicacode{FullForm}, and \mathematicacode{ToExpression}.

\clearpage
\defpage{completemltemplate}
\begin{mathematicaprogram}
:Evaluate:      BeginPackage["CharacteristicSet`"]
:Evaluate:      CoherentAutoreducedSet::usage = "CoherentAutoreducedSet[          --
    { a, b, ...}, {y1,y2,...}, { s1, s2, ... }, Options ] derives a coherent      --
    autoreduced set of the polynomials a, b, ... in the dependent variables y1,   --
    y2,  ... and independent variables s1, s2, ... . Options is an optional       --
    string denoting options as described in the reference manual for the          --
    characteristic set library of Aldor."

:Evaluate:      Begin["`Private`"]

:Evaluate:      CoherentAutoreducedSetAldor::usage = "CoherentAutoreducedSetAldor --
    [ polys, depVars, indepVars, Options ] derives a coherent autoreduced set of  --
    the polynomials encoded in polys in the dependent variables encoded in        --
    depVars and independent variables encoded in indepVars. Options is a string   --
    denoting options as described in the reference manual for the characteristic  --
    set library of Aldor."

:Begin:
:Function:      _charset_coherentAutoreducedSet
:Pattern:       CoherentAutoreducedSetAldor[ langStr_String, polyStr_String,      --
    varStr_String, depStr_String, optStr_String ]
:Arguments:     { langStr, polyStr, varStr, depStr, optStr }
:ArgumentTypes: { String, String, String, String, String }
:ReturnType:    String
:End:

:Evaluate:      CoherentAutoreducedSet[ poly_List, var_List, dep_List,            --
    opts_String:"" ] := ToExpression[ CoherentAutoreducedSetAldor[                --
    "mathematicafullform", ToString[ poly//FullForm ], ToString[ var//FullForm ], --
    ToString[ dep//FullForm ], ToString[ opts//FullForm ] ] ]

:Evaluate:      End[ ]
:Evaluate:      EndPackage[ ]

extern char * _charset_coherentAutoreducedSet( kcharp_ct langStr, kcharp_ct polyStr, 
  kcharp_ct varStr, kcharp_ct depStr, kcharp_ct optStr); 

int main(int argc, char* argv[])
{
  return MLMain(argc, argv);
}
\end{mathematicaprogram}

%%%%%%%%%%%%%%%%%%%%%%%%%%%%%%%%%%%%%%%%%%%%%%%%%
%%% Local Variables: 
%%% ispell-local-dictionary: "english"
%%% End:



%%%%%%%%%%%%%%%%%%%%%%%%%%%%%%%%%%%%%%%%%%%%%%%%%
%%% Local Variables: 
%%% ispell-local-dictionary: "english"
%%% End:
