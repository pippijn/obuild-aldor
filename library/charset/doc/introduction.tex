\clearpage
\section{Introduction}
\defsec{introduction}

This documentation of \LibCharSet is divided into three parts.

The first part discusses the \LibCharSet library itself. This discussion can be dissected into eight parts: 
\begin{itemize}
\item differential domains,
\item algebraic domains,
\item terms,
\item indeterminates, 
\item order,
\item reductions,
\item algorithms, and
\item miscellaneous.
\end{itemize}


The differential domains part is described in \refsec{dpolys} and shows how differential polynomials can be modelled using \LibCharSet. In \refsec{polys}, the algebraic domain part can be found. This part describes \LibCharSet's efforts for algebraic polynomial rings. The terms part in \refsec{exponents} discusses how terms can be stored. In \refsec{indeterminates} a description of how to model differential indeterminates and extensions of indeterminates are given. This treatment constitutes the indeterminates part of \LibCharSet. By \refsec{orders}, the orders part is given, which discusses the implemented orders for differential indeterminates. \refsec[Section]{reductions} is the reduction part of \LibCharSet. There, the different reductions of \LibCharSet are presented. The seventh part is the algorithms part, which can be found in \refsec{algorithms}. This part illustrates how to use the implemented algorithms of \LibCharSet. The miscellaneous part can be found in \refsec{misc} and holds those domains not fitting into one of the other parts.

The second part of this documentation describes how \Aldor's \LibCharSet library can be connected to other computer algebra systems.

In \refsec{preparationsinaldor}, the necessary preparations within the \LibCharSet library are discussed. \refsec[Section]{casmaple} illustrates, how the \LibCharSet library can be connected to \Maple. \refsec[Section]{casmathematica} shows, how the \LibCharSet library can be connected to \Mathematica. Finally, \refsec{cascmdline} presents a command line utility to connect to the \LibCharSet library.

The third and last part of this documentation gives instructions on how to install \LibCharSet in \refsec{installation} and gives the application programming interface with a detailed description of every domain, category, and macro of \LibCharSet in \refsec{api}.

