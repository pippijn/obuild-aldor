% Copyright (C) 2005  Research Institute for Symbolic Computation, J.    
% Kepler University, Linz, Austria                                       
%                                                                        
% Written by Christian Aistleitner                                       
%                                                                        
% This program is free software; you can redistribute it and/or          
% modify it under the terms of the GNU General Public License version 2, 
% as published by the Free Software Foundation.                          
%                                                                        
% This program is distributed in the hope that it will be useful,        
% but WITHOUT ANY WARRANTY; without even the implied warranty of         
% MERCHANTABILITY or FITNESS FOR A PARTICULAR PURPOSE.  See the          
% GNU General Public License for more details.                           
%                                                                        
% You should have received a copy of the GNU General Public License      
% along with this program; if not, write to the Free Software            
% Foundation, Inc., 51 Franklin Street, Fifth Floor, Boston,             
% MA  02110-1301, USA.                                                   
%
\pagebreak
\section{Overview}

\projectname is a general toolkit for input and output functionality in \Aldor.

\LibAldor and \LibAlgebra already provide categories and domains for input and output. \projectname tries to extend these concepts.

\projectname is separated into five parts. These parts are

\begin{itemize}
\item character codes,
\item computer algebra systems,
\item exceptions,
\item data structures, and
\item miscellaneous.
\end{itemize}

All of these parts are treated individually in their corresponding sections.



