% Copyright (C) 2005  Research Institute for Symbolic Computation, J.    
% Kepler University, Linz, Austria                                       
%                                                                        
% Written by Christian Aistleitner                                       
%                                                                        
% This program is free software; you can redistribute it and/or          
% modify it under the terms of the GNU General Public License version 2, 
% as published by the Free Software Foundation.                          
%                                                                        
% This program is distributed in the hope that it will be useful,        
% but WITHOUT ANY WARRANTY; without even the implied warranty of         
% MERCHANTABILITY or FITNESS FOR A PARTICULAR PURPOSE.  See the          
% GNU General Public License for more details.                           
%                                                                        
% You should have received a copy of the GNU General Public License      
% along with this program; if not, write to the Free Software            
% Foundation, Inc., 51 Franklin Street, Fifth Floor, Boston,             
% MA  02110-1301, USA.                                                   
%
\pagebreak
{\huge Although the \projectname environment (its scripts and  Makefiles) itself is ready for use, the documentation of \projectname is still very incomplete.}

Although the documentation of \projectname is far from being completed, \projectname has been released, as other libraries (e.g.: \LibAldorUnit\cite{AldorUnit} and \LibExtIO\cite{ExtIO}) use it and reference to it. \projectname is ready to be used by other \Aldor projects as well, however, \projectname's documentation is still missing.
\pagebreak
\section{Motivation}
\defsec{motivation}


\projectname is serves as model for \Aldor libraries.

The main purpose of \projectname is {\it not} to deliver the functionality of the \projectname's library.

The main purpose of \projectname is to provide a framework for implementing libraries in \Aldor.

The framework of \projectname provides support for
\begin{itemize}
\item building libraries,
\item building unit tests,
\item generating documentation,
\item configuring a library on a foreign system, and
\item installing a library on a foreign system.
\end{itemize}

With the help of \LibModel, developers only have to write code, unit tests, and documentation. Developers do {\it not} have to deal with tracking dependencies or extracting the API of source files.
